% !TEX root = ../0_main/00_main.tex
In the past twenty years several events disrupted global economics and social well-being and generally shook the confidence in the stability of western societies. Popular examples are, the financial crisis, bankruptcy of multiple developed states, populism, war and climate refugees or Brexit. With this background we aimed to identify drivers of societal instability or even collapse. For this purpose a model was developed inspired by the theory of the collapse of complex societies. A simple network model simulated the development of complexity in terms of an administration body as a response to stresses affecting the productivity of the network agents.

We were able to illustrate societal collapse as a function of complexity measured in the share of administration in a network. Furthermore, we identified minimum requirements of the administration and the societal network topology to improve well-being of the society, estimated in terms of produced energy per capita. Finally we provide a mechanism for improving well-being and survival of the modeled society by enabling agents to randomly change between labor and administration, which is effective at very low rates.


% With no doubt, contemporary societies are complex in many dimension
