% !TEX root = ../0_main/00_main.tex
\section{Model description}


\subsection{Tainter inspired network of a runaway society steering into deminishing marginal returns and collapse}
The object of our study is a simple society that has access to an arbitrary resource $a$, which is a placeholder for any sort of energy source.

The society has N members, represented by the nodes of an undirected network. The complexity of the society is modelled by hierarchical societal classes. Each node can be a labourer $L$, an administrator $A$, or a coordinated Labourer $L_c$, which simply means being neighbour of an administrator. A high administration share is regarded as a high complexity.

Only members of $L$ and $L_c$ contribute to the use of $a$, which the society needs for survival. Administrators increase the efficiency of their labouring neighbours (who are therefore members of $L_c$) but do not contribute to the production/survival directly. The energetic output of the society is measured as Energy per capita, $E_{\text{cap}}$:

\begin{equation}
    E_{\text{cap}} = a \ \frac{|L| + |L_c|^e}{N},
\end{equation}

where $e$ is the efficiency increase exponent for neighbours of administrators. The society has an interest in maintaining the energy output above a certain threshold $T$.

In the basic model version, $a$ (initially set to 1) is reduced by a percentage stress $s$ at each time step, $a_{\text{n+1}} = (1-s)\  a_{\text{n}}$.

$s$ represents what Tainter calls \textit{problems to solve}, such as population pressure or changing environmental conditions. If $E_{\text{cap}} < T$, the network reacts by adding another node to $A$. The first administrator is chosen randomly from all nodes with highest degree. Every following administrator will be chosen randomly out of the set of $L_c$ with highest degree.
If $E_{\text{cap}}$ falls under a set death energy level, the simulation ends.

First, the general behaviour of the model was examined using a Barabási–Albert preferential attachment network and compared to the predictions of Tainter's theory. For this, we distinguished between two different assumptions concerning the development of hierarchy in a society: The first is that a society chooses to form an administration class in order to increase the energy output to a higher level, without any external pressure (and continues this strategy). This means that $s = 0$ and $T>1$. The second presumption, which is used by Tainter, says that a society has no choice but to increase complexity in order to solve problems; therefore, a constant stress rate $s > 0$ enforces a reaction even without high ambitions of the society.

In a second part of analysis, the influence of the most important parameters was scanned, taking into account the following network types:

\begin{table}[h!]
\centering
\begin{tabular}{l|r}
Network type & Parameters \\\hline
Barabási–Albert & $m$ (number of edges to attach from a new node to existing nodes) \\
Watts–Strogatz & $k$ (number of nearest neighbours), $r$ (rewiring probability) \\
Erdős-Rényi & $p$ (probability for edge creation)
\end{tabular}
%%\caption{\label{tab:widgets}An example table.}
\end{table}

To ensure comparability, the parameters were chosen such that the mean degree is identical.
In a third step, the effect of a simple model extension was studied, which is the introduction of a population development mechanism.

\subsection{random class exploration as a countermeasure to collapse on a individual basis}


\subsection{Analytic approximation to the mechanistic probabilistic models}
