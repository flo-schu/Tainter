% !TEX root = ../0_main/00_main.tex

\section{Tainter today}

% - our model does not show collapse in tainter's view. Because the complex hierarchies are maintained even during the decrease of energy production (= collapse in our view). According to tainter collapse is manifest in:
% \begin{enumerate}
%   \item lower degree of stratification and social differentiation
%   \item less economic and occupational specialization
%   \item less centralized control
%   \item less behavioral control and regulation
%   \item less investment into complexity (architecture, culture)
%   \item less flow of information
%   \item less sharing, trading and redistribution of resources
%   \item less overall coordination and organization
% \end{enumerate}
%
% however this is also not the aim of the research. It was rather the aim to illustrate a simple mechanism of a many like size of society, distinctiveness of parts, distinct social personalites, specialized roles, mechanisms of organization).
%
% Tainter favours economic explanations as the superior theories of collapse of complex societies. The main themes are (a) decreasing advantages of complexity (b) increasing disadvantages of complexity (c) increasing costliness of complexity. In our model we implemented a and c.
%

1. Observations of heavy administration bodies and bureaucracy in contemporary societies and associated problems
% https://www.researchgate.net/profile/David_Steensma/publication/259608225_Impact_of_Cancer_Research_Bureaucracy_on_Innovation_Costs_and_Patient_Care/links/54b42e500cf2318f0f96bfbf/Impact-of-Cancer-Research-Bureaucracy-on-Innovation-Costs-and-Patient-Care.pdf

more


2. Theory of tainters marginal productivity
Economic explanation for collapse
diminishing marginal returns
(a) reduced advantages of complexity
(b) increased costliness of complexity
((c) increasing disadvantages of complexity)



3. Combination of economic theory and runaway model of a society which will not change course. (Positive feedback loops)
