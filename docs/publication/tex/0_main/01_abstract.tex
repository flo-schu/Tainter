% !TEX root = ../0_main/00_main.tex
In the past twenty years several events disrupted global economics and social wellbeing and generally shook the confidence in the stability of western societies. Popular examples are, the financial crisis, bankrupcy of multiple developed states, populism, war and climate refugees or Brexit. With this background we aimed to identify drivers of societal instability or even collapse. For this purpose a model was developed inspired by the theory of the collapse of complex societies. A simple network model simulated the development of complexity in terms of an administration body as a response to stresses affecting the productivity of the network agents. While initially the development of an administration positively affected the wellbeing of the society, quickly marginal returns were diminished and resulted in the collapse of the society. In order to investigate whether simple mechanisms exist, which could increase the survival and wellbeing of such societies, the effect of random status change (administration, laborer) of individual nodes was implemented, which even at very low levels dramatically increased survival and wellbeing of societies.


% With no doubt, contemporary societies are complex in many dimension
