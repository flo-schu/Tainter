% !TEX root = ../0_main/00_main.tex

\section{Introduction} %Tainter today

How can the net amount of entropy of the universe be massively decreased? (Asimov, 1956). It cannot or at least we do not yet know enough about the universe to tell. Referring to the collapse of societies and the energy death of the universe are more related than one may imagine. Humanity as a society is confined by the availability of energy. At the moment massive amounts of solar energy, which have been stored in the Earths crust are used up at ever increasing rates. Not only does it affect our climate system, but it brings us to another severe problem. Due to the amounts of energy humanity has liberated in the past two centuries since the industrial revolution, humanity was very successful to create a hyper-complex civilization of interdependencies, which brings particularly western societies an unprecedented comfort, life-expectancy and wellbeing.

National states with a high degree of regularization and control emerged from simpler systems with less control systems and higher uncertainty for the individual. Comparing the fraction of the population working in labor directly related to food and agriculture, i.e. labor necessary for survival such as energy provision, of around two hundred years ago with the corresponding fraction of today, a sharp decrease particularly in western nation states is visible. At the same time the fraction of population working in administrative tasks (business, government) has increased multiple times \cite{??}. Undoubtedly this shift is desirable since work in higher levels of control brings advantages for the individual and lower percentages of the population need to work in the tedious and physically demanding labor of the agricultural sector. However, drawing on observations from past civilizations, such complexity brings with it the risk of societal collapse.

Collapse itself is defined as the rapid breakdown of organization, control, information flow, trading and sharing of resources \cite{Tainter.2011}. All these are traits and benefits of complex systems.
Therefore, a prerequisite for collapse of a system is its complexity.
\textcolor{red}{[Jobst: das klingt, als koennten nicht-komplexe Systeme auch nicht kollabieren... Wollen wir das sagen?]}
Furthermore, it is suggested that solving problems leads to complexity, as organization of a social group leads to higher necessities of information flow, the implementation of an invention leads to the need for new shipment routes of resources and manufactured products or the maintenance of an administration and the enforcement of its rules and regulations. According to \cite{Tainter.2011} such increases in complexity always come at a cost of increased time, money, labour, etc., which are all substitutes for energy. This in turn can pose another problem, which again needs to be solved. Tainter calls this the energy-complexity-spiral. Oscar Wilde frames it:
\begin{quote}
    The bureaucracy is expanding to meet the needs of the expanding bureaucracy.
\end{quote}
The aim of our work was to create a very simple, stylized model of a society which has just enough features to display this kind of behaviour. For this we created a network of workers which face external shocks and react to this by increasing its complexity, by transforming workers into administrators. We chose this as the object of our study in order to demonstrate Tainter's concept of complexity and collapse as interdependent relations between energy demands of a society and energy cost of an administration body as a very plastic feature of complexity.
% \textcolor{red}{[Jobst: Aber wir haben doch gar keine zusaetzlichen Hierarchien. Von Anfang an gibt es genau drei Gruppen, es wird nicht weiter verzweigt.]}
% \textcolor{red}{[Jobst: siehe oben.]}

\textcolor{red}{[Jobst: manches von dem nachfolgend auskommentierten scheint mir sinnvoll.]}
% - our model does not show collapse in tainter's view. Because the complex hierarchies are maintained even during the decrease of energy production (= collapse in our view). According to tainter collapse is manifest in:
% \begin{enumerate}
%   \item lower degree of stratification and social differentiation
%   \item less economic and occupational specialization
%   \item less centralized control
%   \item less behavioral control and regulation
%   \item less investment into complexity (architecture, culture)
%   \item less flow of information
%   \item less sharing, trading and redistribution of resources
%   \item less overall coordination and organization
% \end{enumerate}
%
% however this is also not the aim of the research. It was rather the aim to illustrate a simple mechanism of a many like size of society, distinctiveness of parts, distinct social personalites, specialized roles, mechanisms of organization).
%
% Tainter favours economic explanations as the superior theories of collapse of complex societies. The main themes are (a) decreasing advantages of complexity (b) increasing disadvantages of complexity (c) increasing costliness of complexity. In our model we implemented a and c.
%

% \begin{itemize}
%     \item Observations of heavy administration bodies and bureaucracy in contemporary societies and associated problems
% % https://www.researchgate.net/profile/David_Steensma/publication/259608225_Impact_of_Cancer_Research_Bureaucracy_on_Innovation_Costs_and_Patient_Care/links/54b42e500cf2318f0f96bfbf/Impact-of-Cancer-Research-Bureaucracy-on-Innovation-Costs-and-Patient-Care.pdf
% \item also give examples from the past as they link to Tainter's work. quote:
% “The bureaucracy is expanding to meet the needs of the expanding bureaucracy.” ― Oscar Wilde.
%     \item Introduce Tainters theory of marginal productivity
%     \item Economic explanation for collapse: diminishing marginal returns
%     \item Possible effects:
%     \begin{itemize}
%         \item reduced advantages of complexity
%         \item increased costliness of complexity
%         \item (increasing disadvantages of complexity)
%     \end{itemize}
%     \item Combination of economic theory and runaway model of a society which will not change course. (Positive feedback loops)
% \end{itemize}

% Some bullet points by MW (synthesis from some of Tainters works, partly based on this talk: https://www.youtube.com/watch?v=G0R09YzyuCI):

% General introduction:
% \begin{itemize}
%     \item collapse is the rapid simplification of a society (i.e., transition from Roman Empire to the Dark Ages)
%     \item complexity is a state that leads to collapse
%     \item increase in complexity is characterized by a change from small societies with few distinctions to societies that are large and differentional (e.g., Hunter-Gatherer with a few dozen roles to modern societies with 10000-20000 roles)
%     \item social complexity is thus defined as (i) a differentiation and specialization in structure and function and (ii) an increasing control of behaviour (MW: we basically model both of these aspects by (i) introducing the special role of an administrator that (ii) has some control over the labourers)
%     \item without organization there is no complex system. In other words only structure and organization make a complex system and complexity itself serves to regularize these systems
%     \item collapse is defined by a loss of an established level of complexity
%     \item examples of collapsed societies
%     \begin{itemize}
%         \item Western Roman Empire (5th century AD) followed by Dark Ages
%         \item Collapse of Maya (9th century AD)
%         \item Minoan Civilization of Crete (14th century BC)
%         \item Western Zhou Dynasty of China (771 BC)
%     \end{itemize}
% \end{itemize}

% How complexity relates to energy (important for our model):
% \begin{itemize}
%     \item complexity always has a cost, such as calories, time, or money, all of which can be reduced to "energy" (today: fossil fuels)
%     \item complexity helps solving problems (e.g., by inventing new things or creating new positions --> both lead to differentiation, see above) and hence it increases over time
%     \item increased complexity hence also increases the control over human behaviour (i.e., by creating an administration)
%     \item Tainter calls this the "energy-complexity"-spiral (MW: I really like this term!): A larger demand or reduced access to energy causes societies to increase their complexity at the cost of an even higher demand for energy (and so on)
% \end{itemize}

% Tainters thoughts on sustainability (and therefore also resilience):
% \begin{itemize}
%     \item sustainability is an active process of problem solving in order to sustain an established level of welfare, productivity, complexity and so on
%     \item hence, a society's ability to cope with new problems can be regarded as its resilience to external shocks (this is not explicitly stated by Tainter, but we could frame it this way) (\textcolor{blue}{FS: We could even expand this further in my view. If - in Tainter's terms - a society's sustainable capacity is to sustain an established level of complexity then, a resilient society is able to cope with new challenges in a way, which does not increase the complexity of this society.}
%     \item this also implies that societies become vulnerable to collapse through the sheer process of solving problems (\textcolor{blue}{FS: unless the society is able to achieve problem solving without increasing complexity. Maybe by simply reorganizing itself.}
%     \item while the cost of solving a single problem is acceptable, the cumulative costs are the real cause of damage (as in our model, where one single shock alone does not cause a collapse of the society)
%     \item once a certain energy demand is reached a society can already be destroyed by the cost of sustaining itself (again see above for the energy-complexity spiral). In other words, a society's resilience decreases over time (with increasing complexity).
% \end{itemize}
% %more
